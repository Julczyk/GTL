\documentclass{article}

\usepackage[letterpaper,top=2cm,bottom=2cm,left=3cm,right=3cm,marginparwidth=1.75cm]{geometry}

\usepackage{amssymb}
\usepackage{amsthm}
\usepackage[intlimits]{amsmath}
\usepackage[polish]{babel}
\usepackage[utf8]{inputenc}
\usepackage[T1]{fontenc}
\usepackage{xcolor}

\usepackage{array}
\usepackage{multicol}
\usepackage{indentfirst}
\usepackage{graphicx}
\usepackage{subfig}
\usepackage{mathptmx}
\usepackage{geometry}
\usepackage{wrapfig}
\usepackage{float}
\usepackage{enumitem}
\usepackage{caption}
%\captionsetup{labelfont=bf}
%\captionsetup{format=hang}
\usepackage{cancel}
\usepackage{wrapfig}
\usepackage{multirow}
\usepackage{fancyvrb}
\usepackage{emoji}


\title{GTL - The Green Text Programming Language}
\author{Jakub Kierznowski, Mateusz Kotarba, Juliusz Kociński}

\begin{document}
\maketitle

%\begin{abstract}
%Brak.
%\end{abstract}

\section{Wstęp}
Green Text Programming Language (GTL) to uniwersalny, imperatywny język programowania, którego głównym założeniem jest składnia możliwie zbliżona do Greenext - formatu krótkich, charakterystycznie pisanych historyjek internetowych, spopularyzowanego przez portal 4chan.org, a następnie różne fora. W Polsce szerzej jest znany jako klasyczny format pasty (i zapisywany czerwonym tekstem).

\section{Komentarze}

\begin{itemize}
    \item Każda linia musi zaczynać się od znaku \texttt{>} poprzedzonym spacją. Jest ona traktowana jako kod. Po tym znaku można dać dowolną ilość spacji lub tabów, aby uczynić program bardziej czytelnym.
    \item Linia, która nie zaczyna się od \texttt{ >}, jest traktowana jako komentarz.
    \item Komentarze w tej samej linijce muszą być poprzedzone spacją i znakiem \texttt{\#}.
\end{itemize}


\textbf{Składnia:}
\begin{Verbatim}[commandchars=\\\{\}]
\textcolor{green}{ > \{To jest kod\}}
\textcolor{red}{To jest komentarz}
\textcolor{green}{ > \{To jest kod\}} \textcolor{red}{# To jest komentarz w linii}
\end{Verbatim}

\section{Zmienne}

\begin{itemize}
    \item Zmienne muszą być deklarowane z podaniem typu: \texttt{type nazwa\_zmiennej}.
    \item Nazwy zmiennych mogą być dowolnym znakiem z unicode (nawet może być to emoji), z wyjątkiem \texttt{'} i \texttt{,}. Zmienna nie może zaczynać się od cyfry. Zmienna nie może być słowem kluczowym.
    \item GTL obsługuje szereg typów prostych:

    \begin{table}[H]
    \centering
    \begin{tabular}{ccc}
        int                   & → & see   \\
        double                & → & taste \\
        string                & → & hear  \\
        boolean               & → & smell \\
        struktury             & → & spot
    \end{tabular}
    \end{table}

        \item Przypisanie do zmiennej odbywa się za pomocą słowa kluczowego \texttt{is} po czym następuje wyrażenie, lista wyrażeń lub wywołanie funkcji.

\textbf{Składnia definicji zmiennej i przypisania zmiennej:}
{\color{green}
\begin{Verbatim}
> {typ zmiennej} {nazwa zmiennej}
> {typ zmiennej} {nazwa zmiennej} is {wywołanie funkcji, wyrażenie, lista wyrażeń}
> {nazwa zmiennej} is {wywołanie funkcji, wyrażenie, lista wyrażeń}
\end{Verbatim}
}

\newpage

\textbf{Przykład:}
{\color{green}
\begin{verbatim}
 > see baddie is 9
 > taste lunch is someone elses dinner
 > smell flowers is c:
 > flower is :c
\end{verbatim}
}

    \item Dla typów złożonych:
    \begin{itemize}
        \item \texttt{multiple} definiuje dynamiczną tablicę.
        \item \texttt{about} i potem wyrażenie definiuje tablicę znanym rozmiarze.
        \item \texttt{someone elses} definiuje referencję do zmiennej (nie jest tworzona jej kopia).
    \end{itemize}
    \item Tablicę można inicjować słowem \texttt{is} i podaniem rozdzielonych przecinkami lub \texttt{and} wartości.
    \item Dostęp do tablicy jest przez poprzedzenie nazwy tablicy numerem elementu (zaczynając od 0), z dodatkiem \texttt{'th}. Przed \texttt{'th} może się pojawić tylko pojedyncza nazwa lub liczba całkowita.
    \textbf{Składnia dostępu i tworzenia tablicy:}
{\color{green}
\begin{Verbatim}
> {typ zmiennej} multiple {nazwa zmiennej}
> {typ zmiennej} about {wyrażenie} {nazwa zmiennej}
> {typ zmiennej} multiple {nazwa zmiennej} is {wyrażenie}
> and {wyrażenie}, {wyrażenie}
> {całkowity literał, pojedyncza nazwa}'th {nazwa tablicy}
\end{Verbatim}
}

\textbf{Przykład dostępu:}
{\color{green}
\begin{verbatim}
 > see multiple baddies
 > see about 5 men is 1,2,3,4,5
 > see man is 4'th men
 > see best is 2
 > man is best'th men
\end{verbatim}
}
    \item Zmiana typu zmiennej "castowanie" odbywa się przez przypisanie do nowej zmiennej, zmiennej o innym typie.
    \item Przy operacji dodawania z zmienną typu \texttt{hear} (string), pozostałe zmienne są zamieniane na \texttt{hear} (string) i konkatenowane.

    \textbf{Składnia zmiany typu zmiennej:}
    {\color{green}
    \begin{Verbatim}
    > {typ zmiennej} {nazwa zmiennej} is {literał, funkcja, zmienna, wyrażenie (o innym typie)}
    \end{Verbatim}
    }

    \textbf{Przykład:}
    {\color{green}
    \begin{verbatim}
    > taste baddie is 9.5
    > see girl is baddie
    > hear boy is joined by girl
    \end{verbatim}
    }
\end{itemize}

\newpage

\section{Definiowanie struktur (typów)}

Aby stworzyć nową strukturę/typ, należy użyć poniższej składni:

\begin{itemize}
    \item \texttt{> look around} rozpoczyna definicję struktury.
    \item \texttt{> struct\_name} to nazwa struktury.
    \item Następnie definiujemy zmienne, funkcje, pętle wewnątrz struktury.
    \item \texttt{> lose interest} oznacza zakończenie definicji struktury.
    \item Aby dostać się do zawartości pól zmiennych należy podać ich nazwę, a potem napisać \texttt{'s } i nazwę pola.
    \item Nazwy struktur muszą być poprawnymi nazwami (patrz nazwa zmiennej).
    \item Deklaracja struktury utrzymuje się w całym programie.

\end{itemize}
\textbf{Składnia struktury:}
{ \color{green}
\begin{verbatim}
 > look around
 > {nazwa_struktury}
 > {typ_zmiennej} {nazwa_zmiennej}
 > {deklaracja funkcji}
 .
 .
 > lose interest
 >
 > spot {nazwa_struktury} {nazwa_zmiennej}
 > {nazwa_zmiennej}'s {nazwa_pola} is {wyrażnie, funkcja}
 > {inna_zmienna} is {nazwa_zmiennej}'s {nazwa_pola}
\end{verbatim}
}
\textbf{Przykład:}
{ \color{green}
\begin{verbatim}
 > look around
 > Human
 > see age
 > hear name
 > smell alive
 > lose interest

 > spot Human Anon
 > Anon's age is 18
 > see number is Anon's age
\end{verbatim}
}

\section{Zasięg zmiennych}

\begin{itemize}
    \item Zmienna obowiązuje do momentu napotkania pustej linii.
    \item Po napotkaniu pustej linii zmienne zdefiniowane powyżej przestają być dostępne.
    \item Nie stosuje się to do deklaracji funkcji i struktur. Ale sama zdefiniowana struktur (z danymi) już jest usuwana.
\end{itemize}

\textbf{Przykład:}
{\color{green}
\begin{Verbatim}[commandchars=\\\{\}]
> see baddie
\textcolor{red}{tutaj jest komentarz}
> spit baddie  \textcolor{red}{# baddie wciąż istnieje}
\textcolor{red}{drugi komentarz}

> spit baddie  \textcolor{red}{# błąd! baddie nie istnieje, ponieważ wystąpiła pusta linijka}
\end{Verbatim}
}
\section{Operacje logiczne}

\begin{itemize}


\item GTL zawiera wartości logiczne:

\begin{table}[H]
\centering
\begin{tabular}{ccc}
false          & → & :c   \\
true           & → & c:
\end{tabular}
\end{table}


\item GTL zawiera operacje logiczne do porównywania zmiennych:

\begin{table}[H]
\centering
\begin{tabular}{ccc}
==              & → & vibe with   \\
!=              & → & doesn't vibe with       \\
\textless{}     & → & beaten by      \\
\textless{}=    & → & doesn't beat   \\
\textgreater{}  & → & beats          \\
=\textgreater{} & → & unbeaten by    \\
\&\&            & → & also            \\
||              & → & alternatively  \\
zaprzeczenie    & → & not
\end{tabular}
\end{table}


\item GTL zawiera podstawowe operatory matematyczne:

\begin{table}[H]
\centering
\begin{tabular}{ccl}
X+Y             & → & > X joined by Y                               \\
X+=Y            & → & > X is joined by Y                            \\
X++             & → & > X evolves                               \\
X*Y             & → & > X breeding like Y times                     \\
X*=Y            & → & > X is breeding like Y times                  \\
-Y              & → & > the literal opposite of Y                   \\
X--             & → & > X devolves                               \\
$\frac{1}{Y}$   & → & > flipped Y                                   \\
X\%Y            & → & > X whatever left from Y                      \\
X\%=Y           & → & > X is whatever left from Y
\end{tabular}
\end{table}

\item Podobnie jak w innych językach operator modulo, \textit{++}, \textit{--} można wykonać tylko na liczbach całkowitych.
\item \texttt{evolves devolves} można użyć tylko w osobnej linijce, a nie w wyrażeniu.
Na zmienne tekstowe działa tylko dodawanie, konkatenuje ono oba ciągi znaków.
GTL nie zawiera operatorów odejmowania, ani dzielenia. Zamiast tego trzeba dodawać liczbę przeciwną oraz mnożyć przez odwrotność.\\

\textbf{Przykład:}
{\color{green}
\begin{verbatim}
 > see baddie is 9
 > taste division is 5 breeding like flipped baddie times
 > see subtraction is 7 joined by the literal opposite of baddie
\end{verbatim}
}

\newpage

\item Kolejność wykonywania operatorów:
\begin{enumerate}
    \item flipped
    \item the literal opposite of ; breeding like \{wyrażenie\} times
    \item whatever left from
    \item joined by
    \item \{operatory porówniania\}
    \item not
    \item also
    \item alternatively

\end{enumerate}

\item Operatory na tym samym poziomie są wykonywane od lewej do prawej.
\item Przed operatorami \texttt{also} i \texttt{alternatively} może wystąpić nowa linijka. \\
Dodatkowo w mnożeniu \texttt{breeding like \{wyrażenie\} times} przed oraz po wyrażeniu może wystąpić nowa linijka.

\end{itemize}

\section{Wyrażenia}

\begin{itemize}
    \item Wyrażenie to:
    \begin{itemize}
        \item literał
        \item zmienna
    \end{itemize}
    \item Literał to:
    \begin{itemize}
        \item liczba całkowita (bez znaku)
        \item liczba zmiennoprzecinkowa z kropką jako separator części całkowitej od ułamkowej
        \item ciąg znaków w podwójnych cudzysłowach
        \item :c - oznacza wartość false, c: - oznacza wartość true
    \end{itemize}
    \item Zmienna to:
    \begin{itemize}
        \item nazwa zmiennej
        \item dostęp do zmiennej struktury
        \item dostęp do zmiennej w tablicy
    \end{itemize}
    \item Ważne! W literałach występują tylko cyfry i kropki, a nazwa jest dowolna. Dlatego \texttt{-1000} jest to zmienna o nazwie -1000. Aby zrobić -1000 należy użyć \texttt{the literal opposite of 1000}.
\end{itemize}
\section{Warunki \textit{if}}

GTL ma bardzo intuicyjnie zrobione warunki:

\begin{itemize}
    \item \texttt{> implying \{warunek\}} rozpoczyna warunek \textit{if}.
    \item Następnie podajemy kolejne instrukcje do wykonania w \textit{if}.
    \item \texttt{>or} \{warunek\} rozpoczyna część \textit{else if}.
    \item Następnie podajemy kolejne instrukcje do wykonania w \textit{else if}.
    \item \texttt{> or not} rozpoczyna część \textit{else}.
    \item Następnie podajemy kolejne instrukcje do wykonania w \textit{else}.
    \item \texttt{> or sth} kończy \textit{if}'a.

\end{itemize}
\textbf{Składnia if'a:}
{ \color{green}
\begin{verbatim}
 > implying {wyrażenie}
 .
 .
 >or {wyrażenie}
 .
 .
 >or not
 .
 .
 >or sth
\end{verbatim}
}
\textbf{Przykład:}
{ \color{green}
\begin{verbatim}
 > implying anon beaten by max
 > spit max
 > or anon beaten by jessica
 > spit jessica
 > or not
 > spit anon
 > or sth
\end{verbatim}
}

\section{Pętle i iteratory}

Aby stworzyć pętle \textit{while} należy użyć poniższej składni:

\begin{itemize}
    \item \texttt{> think that} rozpoczyna pętlę \textit{while}, po której znajduje się warunek.
    \item Następnie podajemy kolejne linijki zawartości pętli.
    \item \texttt{> reconsider} oznacza konice pętli.
\end{itemize}

\textbf{Składnia pętli while:}
{ \color{green}
\begin{verbatim}
 > think that {warunek}
 .
 .
 > reconsider
\end{verbatim}
}

\textbf{Przykład:}
{ \color{green}
\begin{verbatim}
 > see anon is 0
 > see jessica is 6
 > think that anon doesn't vibe with jessica
 > anon is joined by 2
 > spit anon
 > reconsider
\end{verbatim}
}

\newpage

\section{Funkcje}

\begin{itemize}
    \item Funkcje są definiowane za pomocą słowa kluczowego \texttt{be}.
    \item Nazwa funkcji musi być nazwą (patrz zmienne).
    \item Typ zwracany musi być określony w następnej linii. Zmienna przechowująca wartość zwracaną jest wtedy tworzona. Typ jest określany przez podanie typu i dodanie końcówki \texttt{int}. Jeśli nie jest określony typ zwracany funkcja niczego nie zwraca.
    \item Argumenty funkcji są podawane po słowie kluczowym \texttt{likes}, a wiele argumentów oddziela się przecinkiem lub słowem \texttt{and}. Typ jest określany przez podanie typu i dodanie końcówki \texttt{int}. W przypadku pominięcia funkcja nie przyjmuje żadnych argumentów.
    \item Jeśli chcemy przenieść argumenty przenieść do następnej linijki trzeba zastosować słowo kluczowe \texttt{and}. \texttt{and} piszemy na początku przeniesionej linijki.
    \item Przekazywanie przez referencję: przed typem argumentu należy użyć słowa \texttt{someone elses}.
    \item Funkcja musi kończyć się słowem \texttt{profit}.
    \item Funkcja może mieć każdą nazwę (podobnie jak zmienna), z wyjątkiem zarezerwowanej nazwy \texttt{me}. Funkcja o nazwie \texttt{me} oznacza \texttt{main()}, która jest wykonywana domyślnie przy uruchomieniu programu.
    \item Aby wywołać funkcje trzeba użyć słowa \texttt{call nazwa\_funkcji}. Aby wywołać funkcję z przypisaniem do zmiennej należy użyć słowa \texttt{calling}. Aby wywołać funkcję z argumentami należy po jej nazwie podać słowo \texttt{regarding}, po tym rozdzielone przecinkiem lub słowem \texttt{and} argumenty.
    \item Funkcje nie zagnieżdżone istnieją w całym programie. Funkcje zagnieżdżone są dostępne tylko w danym poziomie zagnieżdżenia.
    \item Możliwe jest przeciążenie funkcji. Deklaracja funkcji o tej samej nazwie ale innych argumetach (inny typ lub różna ilość argumentów).
    \item Możliwe jest podanie wartości domyślnych parametrów.
\end{itemize}

\textbf{Składnia definicji funkcji:}
{\color{green}
\begin{Verbatim}[commandchars=\\\{\}]
 > be \{nazwa_funkcji\}
 > \{typ_zwracany\}ing {nazwa_zmiennej}
 > likes \{typ_zmiennej\}ing \{nazwa_zmiennej\}, \{typ_zmiennej\}ing \{nazwa_zmiennej\}
 > and \{typ_zmiennej\}ing \{nazwa_zmiennej\}
 > \{kod_funkcji\}
 > profit
\end{Verbatim}
}

\textbf{Składnia wywołania funkcji:}
{\color{green}
\begin{Verbatim}
> call {nazwa funkcji} regarding {argument}, {argument}
> and {argument}

> {zmienna} is calling {nazwa funkcji} regarding {argument}, {argument}
> and {argument}
\end{Verbatim}
}

\textbf{Przykład:}
{\color{green}
\begin{Verbatim}
 > be Anon
 > seeing money
 > likes seeing someone elses baddie and spotting math equations
 > money is math joined by the literal opposite of baddie
 > baddie is joined by 1
 > profit

 > see baddie is 7
 > see Ivone is calling Anon regarding baddie and the literal opposite of 1000
\end{Verbatim}
}



Funkcja \texttt{Anon} oblicza wartość na podstawie zmiennych \texttt{math} i \texttt{baddie}, a następnie zwraca wartość przez zmienną \texttt{money}.

\section{Importowanie z innych plików}
\begin{itemize}
    \item Importowanie odbywa się słowem kluczowym \texttt{invite}. Po którym występuje nazwa, zagnieżdżona nazwa lub lista nazwa.
    \item GTL kolejność szukania użytej nazwy zmiennej:
    \begin{enumerate}
        \item lokalna historia
        \item globalne historie
        \item zmiennej w invite
        \item pliki o podanej nazwie w \texttt{invite} i w tych plikach zaproszone nazwy struktur
        \item katalogi o podanej nazwie w \texttt{invite} i powtórzenie kroku poprzedniego
    \end{enumerate}
    \item \texttt{invite} może przyjmować listę, wtedy z tego pliku będą pobrane wskazane elementy
    \item Zaproszone pliki są traktowane jak struktury.
    \item Interpreter zapamiętuje zaproszone już pliki, unika to błędu cyrkulacji \texttt{invite}.
    \item Zapraszane mogą być pliki, struktury, funkcje lub inne zmienne.
\end{itemize}

\textbf{Składnia importu:}
{ \color{green}
\begin{verbatim}
 > invite {zmienna}, {zmienna}
 > and {zmienna}
 > invite {plik}'s {zmienna}, {zmienna}
\end{verbatim}
}
\textbf{Przykład:}
{ \color{green}
\begin{verbatim}
 > invite basia's cake, fake
 > and bake
 > invite baker's bread's flour
\end{verbatim}
}

\section{Wypisywwanie i wpisywanie}

GTL bardzo prosto robi wczytywanie i wypisywanie danych. Stosując analogię do C++:

\begin{table}[H]
\centering
\begin{tabular}{ccc}
cout          & → & spit   \\
cin           & → & swallow
\end{tabular}
\end{table}

\textbf{Składnia:}
{ \color{green}
\begin{verbatim}
 > see anon
 > swallow anon
 > spit anon
\end{verbatim}
}

\textbf{Przykład:}
{ \color{green}
\begin{verbatim}
 > see anon
 > swallow anon
 > spit anon
\end{verbatim}
}

\newpage

\section{Przykładowe programy}

\textbf{Przykład Hello World:}
{\color{green}
\begin{Verbatim}[commandchars=\\\{\}]
\textcolor{red}{Hello World}
 > be me
 > spit "Hello, world!"
 > profit
\end{Verbatim}
}

\textbf{Przykład fibbonaci:}
{\color{green}
\begin{Verbatim}[commandchars=\\\{\}]
\textcolor{red}{This program spits out given ammount of numbers of the fibonacci sequence}
 >be fibonacci
 > like anon
 > see apple is 1
 > see pear is 1
 > see brother is 0
 > think that anon beats brother
 > parity is brother whatever left from 2
 > implying parity
 > spit apple
 > apple is joined by pear
 > or not
 > spit pear
 > pear is joined by apple
 > or sth
 > brother evolves
 > reconsider
 > profit

 > be me
 > see something
 > swallow something
 > call fibonacci regarding something
 > profit
\end{Verbatim}
}




\end{document}